\input{prefix}

\title{\Huge\textbf{{Mathematical modelling in biology}}\\\LARGE Definitions and theorems}

\author{
  Giacomo Fantoni \\
  \small telegram: \href{https://t.me/GiacomoFantoni}{@GiacomoFantoni} \\[3pt]
  \small Github: \href{https://github.com/giacThePhantom/mathematical-modelling-in-biology}{https://github.com/giacThePhantom/mathematical-modelling-in-biology}\\
}


\begin{document}

  \maketitle
  \tableofcontents

\section{Differential equations}
A differential equation that relates a function to its derivative.
They are characterized by the order of the derivative and other criteria, which are useful in determining the approach to a solution.

  \subsection{Ordinary differential equation}
  An ordinary differential equation is a differential equation whose unknown consists of a function:

  $$y(t): \mathbb{R}\rightarrow\mathbb{R}^n$$

  Of one variable $t$ and involves the derivative in $dt$ of that function.
  ODEs have the form:

  $$\frac{dy(t)}{dt} = f(t, y(t))$$

  To check whether a candidate solution is valid it is enough to compute its derivative and check that it is equal to $f(t, y(t))$.

  \subsection{Cauchy problem}
  In general differential equations have infinite solutions, but if we impose an initial condition we can find a unique solution.
  This is the initial value or Cauchy problem, which is in the form:

  $$\begin{cases}
    \frac{dy(t)}{dt} = f(t, y(t))\\
    y(t_0) = y_0
  \end{cases}$$

  \subsection{Autonomous equations}
  A first order ODE iss said to be autonomous if its right hand side does not explicitly depend on $t$.
  It will be in the form:

  $$\frac{dy(t)}{dt} = f(y(t))$$

  Givena a particular solution $y_\alpha(t)$ for a Cauchy problem with $y(0) = y_0$ and another $t_\beta(t)$ for which $t(t_0) = t_0$, then:

  $y_\beta(t) = y_\alpha(t - t_0)$

  \subsection{Separable equations}
  An equation is separable if it can be written in the form:

  $$\frac{dy(t)}{dt} = f(t)g(y(t))$$

  All autonomous equations are separable, but not all separable equations are autonomous.
  Moreover all seaparble ODE with $f(t) = k$ are called constant coefficient problems.

    \subsubsection{Separability}
    Consider a differential equation in the form:

    $$\frac{dy(t)}{dt} = f(t)g(y(t))$$

    Let $F(t)$ be the primitive of $f(t)$ and $H(y(t))$ the primitive of $\frac{1}{g(y(t))}$.
    Then:

    \begin{multicols}{2}
      \begin{itemize}
        \item If $y(t)$ is a solution of $\frac{dy(t)}{dt} = f(t)g(y(t))$ such that $g(y(t))\neq 0$, there exists a constant $c$ such that $H(y(t)) = F(t) + c\ \forall t$.
        \item If $y(t)$ satisfies $H(y(t)) = F(t) + c\ \forall t$ such that $g(y(t))\neq 0$ , then $y(t)$ is a solution of the equation.
      \end{itemize}
    \end{multicols}

  \subsection{Linear ODE}
  A first order linear ODE is in the form:

  $$\frac{dy(t)}{dt} = a(t)y(t) + b(t)$$

  \begin{multicols}{2}
    \begin{itemize}
      \item If $b(t) = 0$ the equation is homogeneous and can be solved by the separation of variables.
      \item If $b(t) \neq 0$ it is non-homogeneous, for which in general the separatio of variables is not effective.
      \item If $a(t) = a\land b(t) = b$ it is autonomous.
      \item If $a(t) = a$ and $b(t)$ any, this becomes a constant coefficient problem.
    \end{itemize}
  \end{multicols}

  \subsection{Direction field}
The direction field allows to graphically fid some properties of a solution of a DE, without explicitly solving it.
The DE tells that if a solution satisfies an initial condition then the slope of the graph of $y(t)$ computed at $t_0$, which is $y'(t_0)$, must be equal to $f(t_0, y_0)$.
Consequently, if in every point $(t_0, y_0)$ a small segment of slope $f(t_0, y_0)$ is drawn, then the solution must be tangent to all of them.

	\subsection{Autonomous equations}
	Autonomous equations will show the same pattern for each $t$.
	So all columns in the cartesian plane will looke the same.

  \subsection{Equilibrium points}
  Given a first order ODE, equilibrium points are particular solutions such that:

  $$\frac{dy(\bar{t})}{dt} = 0$$

  Their derivative is zero for any value of $t$.
  They are constant solutions.

    \subsubsection{Stability}
    The stability of an equilibrium solution is classified according to the behavior of the solutions generated by initial conditions close to the point.
    In particular:

    \begin{multicols}{2}
      \begin{itemize}
        \item An equilibrium $y_e(t)$ is stable if $\forall\epsilon>0\exists U$ neighbourhood of $(t_e, y_e)$ such that $(t_i,y_i)\in U\rightarrow y_i(t)-y_e(t)\le\epsilon\ \forall t$.
          An equilibrium is stalbe if solution arising from initial point close to the initial point remain close to the equilibrium solution.
        \item An equilibrium $y_e(t)$ is asymptotically stable or attractive if, in addition to being stalbe, it is true that:

          $$\lim\limits_{t\rightarrow\infty}y_i(t) = y_e(t)$$

          If solution arising close to the equilibrium converge to it.
        \item An equilibrium $y_e(t)$ is unstable or repulsive if $\exists\eta:\forall \epsilon>0\exists (t_i, y_i)\Rightarrow |(t_e, y_e)-(t_i,y_i)| < \epsilon\land |y_e(t)-y_i(t)|\ge\eta$.
          If there are solutions that diverge from the equilibrium.
      \end{itemize}
    \end{multicols}








\end{document}
