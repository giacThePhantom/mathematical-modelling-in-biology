\chapter{Introduction to ordinary differential equations}

\section{General concepts and methods}

	\subsection{First order differential equations}
	A first order differential equation is written in normal form as:

	$$y'(t) = f(t, y(t))$$

	Where $f$ is a function of variables $t$ and $y$.

		\subsubsection{Autonomous differential equations}
		In the case that $f$ does not depend on the variable $t$ the equation is said to be autonomous: the law that regulates how $y(t)$ changes does not depend on time.

		\subsubsection{Solution of a differential equation}
		The solution of a differential equation if a function $y(t)$.
		To check whether $y(t)$ is a solution it is sufficient to compute its derivative and check that it is equal to the right hand side of the equation.
		So a solution if a function $y:I\rightarrow\mathbb{R}$ where $I$ is an interval of $\mathbb{R}$ sich that $y'(t) = f(t,y(y))\forall t\in I$.

		\subsubsection{Cauchy problems}
		A differential equation has infinite solutions as a function has infinite primitives.
		In order to predict the future evolution of a quantity it is necessary to know, besides the law regulating its dynamics or differential equation, an initial condition.
		An initial value problem or Cauchy problem is constituted by a differential equation and an initial condition:

		$$\begin{cases} y'(t) = f(t,y(t))\\y(t_0) = y_0\end{cases}$$

		A solution to an initial value problem is a function defined on an interval $I$ that contains the point $t_0$ such that for every $t\in I, y'(t) = f(t, y(t)))$ and also it is true that $y(t_0) = y_0$.

	\subsection{Direction fields}
	It is usually difficult or impossible to find exact solutions of differential equations.
	Because of this it is necessary to settle for something less complete.
	Direction fields allow to perform a graphical study of the qualitative behaviour of the solutions.
	The idea sit that according to the equation $y'=f(t,y)$ if a solution satisfies $y(t_0) = y_0$, the slope of the graph of $y(t)$ calculated at $t_0$, which is by definition $y'(t_0)$ must be equal to $f(t_0, y_0)$.
	Consequently, if in every $(t_0, y_0)$ a small segment of slop $f(t_0, y_0)$ is traced the slope of a solution that goes through that point is indicated.

		\subsubsection{Draying a direction field}
		Many point $(\bar{t}, \bar{y})$ in a Cartesian plane are chosen an in each of them a small segment of slope $f(\bar{t}, \bar{y})$ are drawn.
		In this way the field of direction of $y' = f(t, y)$ can be built.
		Then a solution can be drawn making sure it is at all point tangent to the direction field.

		\subsubsection{Autonomous equations}
		Autonomous equations are a class of differential equations in which the right hand side does not depend on $t$:

		$$y' = g(y)$$

		This means that the law that regulates the dynamics of the $y$ variable does not change over time.
		In particular if $y(t)$ is a solution, also $y(t-t_0)$ is a solution for all $t_0$.
		Graphically a solution can be moved horizontally and reach another one.
		Considering direction fields, each column in it will be equal to all the others, so it is sufficient to draw them at a given $t$ and repeat them for all the values of $t$.

	\subsection{Equilibria}
	Given the autonomous differential equation $y' = f(y(t))$ the points $\bar{y}$ such that $f(\bar{y})=0$ are said to be equilibria because if $y(0) = \bar{y}$ then $y(t) = \bar{y}$ for every $t$.
	Since solutions cannot cross each other if $f(y_0)>0$ then $f(y(t))>0$ for every $t$ and then $y'(t) = f(y(t))>0$ or $y(t)$ is an increasing function.
	The same happens in the case of $f(y_0)<0$.

\section{Solution through separation of variables}
