\graphicspath{{chapters/ex1/images}}
\chapter{Problems of week 1}

\section{Problem 1}
Consider the logistic equation:

$$\frac{dN(t)}{dt} = r\left(1-\frac{N(t)}{K}\right)N(t)$$

Where $r$ and $K$ are two positive constants.

  \subsection{Draw the direction field of this equation. From that you should be able to understand the value of $\mathbf{\lim\limits_{t\to\infty}N(t)}$ for any solution such that $\mathbf{N(0) = N_0>0}$}
  This is an autonomous equation, so each column of the direction field will be equal to each other.
  Computing the behaviour of the equation:

  \begin{align*}
    \frac{dN(t)}{dt} &\ge 0\\
    r\left(1-\frac{N(t)}{K}\right)N(t) &\ge 0\\
    N(t)\ge 0\land N(t)\le K
  \end{align*}

  Now, solving the inequality:

  \begin{figure}[H]
    \centering
    \begin{tikzpicture}
      %Vertical lines
      \draw[-latex] (-2,0)--(5,0);
      \draw (0,-2)--(0,.1);
      \node[above] at (0,0.1) () {\small 0};
      \draw (3,-2)--(3,.1);
      \node[above] at (3,0.1) () {\small K};
      \draw (-2,-1.5)--(5,-1.5);

      %First solution
      \node at (-3,-.5) {$N(t)\ge 0$};
      \node at (-1,-.5) () {-};
      \node at (1.5,-.5) () {+};
      \node at (4,-.5) () {+};

      %Second solution
      \node at (-3,-1) {$N(t) \le K$};
      \node at (-1,-1) () {+};
      \node at (1.5,-1) () {+};
      \node at (4,-1) () {-};

      %Final solution
      \node at (-1,-2) {-};
      \node at (1.5,-2) () {+};
      \node at (4,-2) () {-};
    \end{tikzpicture}
  \end{figure}

  From this we get:

  $$\frac{dN(t)}{dt}\ge 0\rightarrow N(t)\in[0,K]$$
  Now the direction field can be drawn.
  Using Matlab:

  \begin{lstlisting}[language=Matlab]
    [T, N] = meshgrid(-2:0.2:4, -2:0.2:4);
    r = 1; K = 2;
    S = r.*(1.-(N./K)).*N;
    L = sqrt(1 + S.^2);
    quiver(T,N, 1./L,S./L, 0.5), axis tight
  \end{lstlisting}

  So that the direction field:

  \begin{figure}[H]
    \centering
    \includegraphics[width=\textwidth/2]{01_direction_field}
  \end{figure}

  \subsection{Consider $\mathbf{u(t) = \frac{1}{N(t)}}$. Show that $\mathbf{u(t)}$ satisfies a linear non-homogeneous equation. By solving the resulting equation for $\mathbf{u(t)}$ find the solution of the equation together with the initial condition $\mathbf{N(0) = N0 > 0}$}
  \textit{This trick is a special case of the trick used for Bernoulli equations; check any text on ordinary differential equations.}\\
  A linear non-homongeneous equation is in the form $\frac{dy(t)}{dt} = a(t)y(t)+b(t)$, with $b(t) \neq 0$.

  So now:

  \begin{align*}
    \frac{dU(t)}{dt} &= \frac{d\frac{1}{N(t)}}{dt}\\
                     &= -\frac{1}{N^2(t)}\frac{dN(t)}{dt}\\
                     &= \frac{1}{N^{\cancel{2}}(t)}r\left(\frac{N(T)}{K}-1\right)\cancel{N(t)}\\
                     &= -\frac{1}{N(t)}r + \frac{1}{\cancel{N(t)}}\frac{r\cancel{N(t)}}{K}\\
                     &= -\frac{1}{N(t)}r + \frac{r}{K}\\
                     &= -rU(t)+\frac{r}{K}
  \end{align*}

  So, in the end $U(t)$ is a linear non-homogeneous equation.
  Now to solve $\frac{dU(t)}{dt}$, we divide it by the factor of the corresponding solution of the homogeneous equation:

  \begin{align*}
    \frac{dQ(t)}{dt} &= -rQ(t)\\
    \frac{dQ(t)}{Q(t)} & = -rdt\\
    \int\frac{1}{Q(t)}dQ(t) &= -\int rdt\\
    \log(Q(t)) &= -rt + C\\
    Q(t) &= e^{-rt + C}\\
    Q(t) &= Ce^{-rt}
  \end{align*}

  So, now consider $W(t) = \frac{U(t)}{e^{-rt}} = U(t)e^{rt}$ and take its derivative:

  \begin{align*}
    \frac{dW(t)}{dt} &= \frac{dU(t)}{dt}e^{rt} + U(t)\frac{de^{rt}}{dt}\\
                     &= \left(-rU(t) + \frac{r}{K}\right)e^{rt} + rU(t)e^{rt}\\
                     &= \cancel{-rU(t)e^{rt}} + \frac{re^{rt}}{K} \cancel{ rU(t)e^{rt}}\\
                     &= \frac{re^{rt}}{K}
  \end{align*}

  Now integrating $\frac{dW(t)}{dt}$:

  \begin{align*}
    W(t) = \int\frac{dW(t)}{dt} &= \int\frac{re^{rt}}{K}dt\\
                                &=\frac{r}{K}\int e^{rt}dt\\
                                &= \frac{r}{K}\left(\frac{1}{r}e^{rt} + C\right)\\
                                &= \frac{\cancel{r}}{K\cancel{r}}e^{rt} + \frac{C}{K}\\
                                &= \frac{1}{K}e^{rt} + \frac{C}{K}
  \end{align*}

  Now, remembering that $W(t) = \frac{U(T)}{e^{-rt}}$:

  \begin{align*}
    U(t) &= W(t)e^{-rt}\\
         &= \left(\frac{1}{K}e^{rt} + \frac{C}{K}\right)e^{-rt}\\
         &= \frac{1}{K}\cancel{e^{rt}}\cancel{e^{-rt}}+\frac{C}{K}e^{-rt}\\
         &= \frac{1}{K}+\frac{C}{K}e^{-rt}
  \end{align*}

  Moreover, $U(t) = \frac{1}{N(t)}$:

  \begin{align*}
    N(t) &= \frac{1}{U(t)}\\
         &= \frac{1}{\frac{1}{K} + \frac{C}{K}e^{-rt}}\\
  \end{align*}

  Now, considering that $N(0) = N_0$:

  \begin{align*}
    N(0) = \frac{1}{\frac{1}{K} + \frac{C}{K}e^{-r0}} &= N_0\\
    \frac{1}{\frac{1}{K} + \frac{C}{K}} &= N_0\\
    \frac{1}{K} + \frac{C}{K} &= \frac{1}{N_0}\\
    \frac{C}{K} &= \frac{1}{N_0} - \frac{1}{K}\\
    C &= \frac{K}{N_0} - 1\\
  \end{align*}

  So, the solution for $N(t)$ is:

  \begin{align*}
    N(t) &= \frac{1}{\frac{1}{K} + \frac{1}{K}\left(\frac{K}{N_0}-1\right)e^{-rt}}\\
         &= \frac{1}{\frac{1}{K} + \left(\frac{1}{\cancel{K}}\frac{\cancel{K}}{N_0}-\frac{1}{K}\right)e^{-rt}}\\
         &= \frac{1}{\frac{1}{K} + \left(\frac{1}{N_0}-\frac{1}{K}\right)e^{-rt}}\\
  \end{align*}

  \subsection{The equation has two equilibria: $\mathbf{N = 0}$ and $\mathbf{N = K}$ (check!). Show that if $\mathbf{r > 0}$ and $\mathbf{K > 0}$, $\mathbf{0}$ is an unstable equilibrium, while $\mathbf{K}$ is asymptotically stable}

  \begin{align*}
    \frac{dN(t)}{dt} &= 0\\
    r\left(1-\frac{N(t)}{K}\right)N(t) &= 0\\
    N(t)= 0\land N(t)= K
  \end{align*}

  An equilibrium $\bar{y}$ of a differential equation in the form $\frac{dy(t)}{dt} = f(t, y(t))$ is asymptotically stable if $\frac{f(t, y(t))}{dt}(\bar{y}) < 0$, if $\frac{f(t, y(t))}{dt}h(\bar{y})>0$, then it is unstable.
  Now computing $\frac{f(t, y(t))}{dt}$:

  \begin{align*}
    \frac{f(t, y(t))}{dt} &= \frac{d}{dt}r\left(1-\frac{N(t)}{K}\right)N(t)\\
                          &=r\left(-\frac{1}{K}\right)N(t) + r\left(1-\frac{N(t)}{K}\right)\\
                          &=-\frac{rN(t)}{K} + r - \frac{rN(t)}{K}\\
                          &= r-\frac{2rN(t)}{K}
  \end{align*}

  So now, considering the equilibrium $N(t) = 0$:

  $$r - \frac{2r0}{K} = r$$

  So $\frac{f(t, y(t))}{dt} = r >0$, the equilibrium is unstable.\\
  Looking at $N(t) = K$:

  $$r - \frac{2r\cancel{K}}{\cancel{K}} = -r$$

  So $\frac{f(t, y(t))}{dt} = -r <0$, the equilibrium is asymptotically stable.

  \subsection{What is wrong (from the modelling point of view, if $\mathbf{N(t)}$ represents population density) in considering the logistic equation with $\mathbf{r < 0}$ and $\mathbf{K > 0}$?}
  A negative $r$ represents a negative growth rate, meaning that the population decreases with time, meaning that the population will never reach the limiting value $K$, but will always settle at $0$.

  \subsection{Change the time unit to $\mathbf{\tau = rt}$ and the variable to $\mathbf{y = \frac{N}{K}}$. Show that $\mathbf{\frac{dy}{d\tau}=y(1-y)}$}
  \textit{This procedure is known as scaling or non-dimensionalizing; we will see more of that in the next lectures.}\\

  Consider that:

  \begin{align*}
    \tau = rt &\Rightarrow t = \frac{\tau}{r}\\
    y(t) = \frac{N(t)}{K} &\Rightarrow N(t) = y(t)K
  \end{align*}

  Meaning that, by the chain rule:

  $$\frac{dN(t)}{dt} = \frac{dKy(\tau)}{d\frac{\tau}{r}} = Kr\frac{dy(\tau)}{d\tau}$$

  Substituting in the equation:

  \begin{align*}
    Kr\frac{dy(\tau)}{d\tau} &= r\left(1-\frac{\cancel{K}y(\tau)}{\cancel{K}}\right)Ky(\tau)\\
    \frac{dy(\tau)}{d\tau} &=  \frac{\cancel{r}\left(1-y(\tau)\right)\cancel{K}y(\tau)}{\cancel{K}\cancel{r}}\\
                           &=(1-y(\tau))y(\tau)
  \end{align*}



\section{Problem 2}
Zeisel et al. considered the following system for the concentration of mRNA.
The variables are $M(t)$, the concentration of mature mRNA, and $P(t)$, the concentration of precursor mRNA:

$$\begin{cases}
  \frac{dP(t)}{dt} = b(t) - a_1P(t)\\
  \frac{dM(t)}{dt} = a_1P(t) - a_2(t)M(t)
\end{cases}$$

Where:

\begin{multicols}{2}
  \begin{itemize}
    \item $b(t)$ is the (gene-specific and time-dependent) production rate.
    \item $a_1$ is the conversion (splicing) rate of pre-mRNA to mRNA.
    \item $a_2(t)$ is the (time-dependent) degradation rate of mRNA.
  \end{itemize}
\end{multicols}

  \subsection{Assume $\mathbf{b(t) \equiv b}$ and $\mathbf{a_2(t) \equiv a_2}$ are constant. Find the equilibria of the system}
  The system becomes:

  $$\begin{cases}
    \frac{dP(t)}{dt} = b - a_1P(t)\\
    \frac{dM(t)}{dt} = a_1P(t) - a_2M(t)
  \end{cases}$$

  Finding the equilibria means making the derivatives equal to $0$:

  \begin{align*}
    \begin{cases}
      b - a_1P(t) &= 0\\
      a_1P(t) - a_2M(t) &=0\\
    \end{cases}\\
    \begin{cases}
      P(t) &= \frac{b}{a_1}\\
      a_1P(t) - a_2M(t) &=0\\
    \end{cases}\\
    \begin{cases}
      P(t) &= \frac{b}{a_1}\\
      \cancel{a_1}\frac{b}{\cancel{a_1}} - a_2M(t) &=0\\
    \end{cases}\\
    \begin{cases}
      P(t) &= \frac{b}{a_1}\\
      M(t) &=\frac{b}{a_2}\\
    \end{cases}\\
  \end{align*}

  So, the equilibria of the systems are:

  $$P(t) = \frac{b}{a_1}\qquad M(t) = \frac{b}{a_2}$$

  \subsection{Under the same assumptions find the explicit solution}
  Solving $\frac{dP(t)}{dt} = -a_1P(t) + b$ first.
  First find the factor to divide $P(t)$ with as the solution of the corresponding homogeneous problem, which is $e^{-a_1t}$.
  Now let $U(t) = \frac{P(t)}{e^{-a_1t}} = P(t)e^{a_1t}$ and computing its derivative:

  \begin{align*}
    \frac{dU(t)}{dt} &= \frac{dP(t)}{dt}e^{a_1t} + P(t)\frac{de^{a_1t}}{dt}\\
                     &= \cancel{a_1P(t)e^{a_1t}} \cancel{- a_1P(t)e^{a_1t}} + be^{a_1t}\\
                     &= be^{a-1t}
  \end{align*}

  Recomputing the primitive via an integral:

  \begin{align*}
    U(t) = \int\frac{dU(t)}{dt} &= \int be^{a_1t}\\
                                &= \frac{b}{a_1}e^{a_1t} +C_1 \\
  \end{align*}

  Now, $P(t) = U(t)e^{-a_1t}$:

  \begin{align*}
    P(t) &= \frac{b}{a_1}\cancel{e^{a_1t}}\cancel{e^{-a_1t}} + C_1e^{-a_1t}\\
         &= \frac{b}{a_1} + C_1e^{-a_1t}\\
  \end{align*}

  Solving now the second, substituting inside the solution of the first: $\frac{dM(t)}{dt} = a_1\left(\frac{b}{a_1} +C_1e^{-a_1t}\right) - a_2M(t)$:

  \begin{align*}
    \frac{dM(t)}{dt} &= \cancel{a_1}\frac{b}{\cancel{a_1}} +a_1C_1e^{-a_1t} - a_2M(t)\\
                     &= b + a_1C_1e^{-a_1t} - a_2M(t)\\
  \end{align*}

  Again find the factor as the solution of the corresponding homogeneous problem, $e^{-a_2t}$.
  Let $Q(t) = \frac{M(t)}{e^{-a_2t}} = M(t)e^{a_2t}$ and compute its derivative:

  \begin{align*}
    \frac{dQ(t)}{dt} &= M(t)\frac{de^{a_2t}}{dt} + \frac{dM(t)}{dt}e^{a_2t}\\
         &= \cancel{a_2M(t)e^{a_2t}} + be^{a_2t} + a_1C_1e^{(a_2-a_1)t} \cancel{-a_2M(t)e^{a_2t}}\\
         &= be^{a_2t} + a_1C_1e^{(a_2-a_1)t}
  \end{align*}

  Recomputing the primitive via an integral:

  \begin{align*}
    Q(t) = \int\frac{dQ(t)}{dt} &= \int\left[be^{a_2t} + a_1C_1e^{(a_2-a_1)t}\right]dt\\
                                &= b\int e^{a_2t}dt + a_1C_1\int e^{(a_2-a_1)t}dt\\
                                &=\frac{b}{a_2}e^{a_2t} + \frac{a_1C_1}{a_2-a_1}e^{(a_2-a_1)t} + C_2
  \end{align*}

  Now, $M(t) = Q(t) e^{-a_2t}$:

  \begin{align*}
    M(t) &= \frac{b}{a_2}e^{a_2t}e^{-a_2t} + \frac{a_1C_1}{a_2-a_1}e^{(a_2-a_1)t}e^{-a_2t} + C_2e^{-a_2t}\\
         &= \frac{b}{a_2}e^{(\cancel{a_2}-\cancel{a_2})t} + \frac{a_1C_1}{a_2-a_1}e^{(\cancel{a_2}-\cancel{a_2}-a_1)t} + C_2e^{-a_2t}\\
         &= \frac{b}{a_2} + \frac{a_1C_1}{a_2-a_1}e^{-a_1t} + C_2e^{-a_2t}\\
  \end{align*}

  So that, in the end the solution for the system are:

  $$\begin{cases}
    P(t) &= \frac{b}{a_1} + C_1e^{-a_1t}\\
    M(t) &= \frac{b}{a_2} + \frac{a_1C_1}{a_2-a_1}e^{-a_1t} + C_2e^{-a_2t}\\
  \end{cases}$$

  \subsection{Assume now $\mathbf{a_2(t) \equiv a_2}$ and $\mathbf{b(t) = \begin{cases} \bar{b} & t < \bar{t}\\0 &t > \bar{t}\end{cases}}$ It can be easily found that $\mathbf{M(t)}$ tends to $\mathbf{0}$ as $\mathbf{t\to\infty}$; at which rate does it decay?}

  Computing the case in which $b(t) = 0$:

  $$\frac{dP(t)}{dt} = -a_1P(t)$$

  Which, solving through the separation of variables:

  \begin{align*}
    \frac{dP(t)}{P(t)} &= -a_1dt\\
    \int\frac{1}{P(t)}dP(t) &= \int -a_1dt\\
    \log(P(t)) &= -a_1t + C_1\\
    P(t) &= e^{-a_1t + C}\\
    P(t) &= C_1e^{-a_1t}\\
  \end{align*}

  Now solving $\frac{dM(t)}{dt} = a_1C_1e^{-a_1t} - a_2M(t)$.
  Again find the factor as the solution of the corresponding homogeneous problem, $e^{-a_2t}$, and introduce $W(t) = \frac{M(t)}{e^{-a_2t}} = M(t)e^{a_2t}$.
  Computing its derivative:

  \begin{align*}
    \frac{dW(t)}{dt} &= \frac{dM(t)}{dt}e^{a_2t} + M(t)\frac{de^{a_2t}}{dt}\\
                     &= a_1C_1e^{(a_2-a_1)t} \cancel{-a_2M(t)e^{a_2t}} + \cancel{a_2M(t)e^{a_2t}}\\
                     &= a_1C_1e^{(a_2-a_1)t}
  \end{align*}

  Recomputing the primitive via an integral:

  \begin{align*}
    W(t) = \int\frac{dW(t)}{dt} &= \int a_1C_1e^{(a_2-a_1)t}dt\\
                                &= a_1C_1\int e^{(a_2-a_1)t}dt\\
                                &=\frac{a_1C_1}{a_2-a_1}e^{(a_2-a_1)t} + C_2
  \end{align*}

  Now, $M(t) = W(t) e^{-a_2t}$:

  \begin{align*}
    M(t) &= \frac{a_1C_1}{a_2-a_1}e^{(a_2-a_1)t}e^{-a_2t} + C_2e^{-a_2t}\\
         &= \frac{a_1C_1}{a_2-a_1}e^{(\cancel{a_2}-\cancel{a_2}-a_1)t} + C_2e^{-a_2t}\\
         &= \frac{a_1C_1}{a_2-a_1}e^{-a_1t} + C_2e^{-a_2t}\\
  \end{align*}

  So that, now $M(t)$ is equal to:

  $$\begin{cases}
    M(t) = \frac{\bar{b}}{a_2} + \frac{a_1C_1}{a_2-a_1}e^{-a_1t} + C_2e^{-a_2t} & t<\bar{t}\\
    M(t) = \frac{a_1C_1}{a_2-a_1}e^{-a_1t} + C_2e^{-a_2t} & t>\bar{t}\\
  \end{cases}$$

  Computing now $\lim\limits_{t\to\infty}M(t)$:

  \begin{align*}
    \lim\limits_{t\to\infty} \frac{a_1C_1}{a_2-a_1}e^{-a_1t} + C_2e^{-a_2t}&= \frac{a_1C_1}{a_2-a_1}e^{-a_1\infty} + C_2e^{-a_2\infty}\\
                                                                           &=\frac{a_1C_1}{a_2-a_1}0 + C_20\\
                                                                           &= 0
  \end{align*}

  The decay rate is the exponential factor of $M(t)$.
  To compute it:

  So that the concentration decays exponentially.

\section{Problem 3}
The thorium-uranium method for dating rocks is based on the fact that Uranium234 decays into Thorium-230 which in turn decays into other elements.
Set $t = 0$ the rock formation time and denoting $U(t)$ [$T(t)$] the amount of Uranium-234 [Thorium-230] in the rock at time $t$ ( measured in years), the following differential equation system is written:

$$\begin{cases}
  \frac{dU(t)}{dt} = -aU(t) \\
  \frac{dT(t)}{dt} = aU(t) - bT(t) \\
  U(0) = U_0\\
  T(0) = 0
\end{cases}$$

Where:

\begin{multicols}{2}
  \begin{itemize}
    \item $a \approx 5.9 \cdot 10^{-6}\frac{1}{years}$
    \item $b \approx 1.9 \cdot 10^{-5}\frac{1}{years}$
    \item $U_0$ is the initial amount of Uranium-234
  \end{itemize}
\end{multicols}



Note that, based on geological principles, it is believed that there was no thorium at the time of rock formation.

  \subsection{Solve the equation for $\mathbf{U(t)}$}

  \begin{align*}
    \frac{dU(t)}{dt} &= -aU(t) \\
    \frac{dU(t)}{dU(t)} &= -adt \\
    \int \frac{q}{dU(t)}dt & \int -adt \\
    \log(U(t)) &= -at + C \\
    U(t) &= e^{-at + C}\\
    U(t) &= C_1e^{-at}
  \end{align*}

  Considering $U(0) = U_0$:

  \begin{align*}
    C_1e^{-a0} &= U_0\\
    C_1 = U_0
  \end{align*}

  So, in conclusion:

  $$U(t) = U_0e^{-at}$$

  \subsection{How can the quantities of $\mathbf{a}$ and $\mathbf{b}$ can be interpreted? From the data provided can we infer the half-live of Uranium-234 and Thorium-230?}

  \begin{multicols}{2}
    \begin{itemize}
      \item $a$ is the rate at which Uranium-234 decays into Thorium-230.
      \item $b$ is the rate at which Thorium-230 decays into other elements.
    \end{itemize}
  \end{multicols}

  To compute the half-life of Uranium-234:

  \begin{align*}
    U(t_1) &= \frac{1}{2}U(t_2)\\
    \cancel{U_0}e^{-at_1} &= \frac{1}{2}\cancel{U_0}e^{-at_2}\\
    2 &= \frac{e^{-a t_2}}{e^{-a t_1}}\\
    e^{-a t_2 + at_1} &= 2\\
    e^{a(t_1 - t_2)} &= 2\\
    a(t_1 - t_2) &= \log(2)\\
    t_1 - t_2 &= \frac{\log(2)}{a}\\
    t_1 - t_2 &= \frac{\log(2)}{5.9 \cdot 10^{-6}} = 117482.6\text{ years}\\
  \end{align*}

  Now, computing the half-life of Thorium-230, in this case the half life depends only on the time and not on the initial concentration, so we can ignore the effect of increase of the Uranium-234.
  So the differential equation becomes $\frac{dT(t)}{dt} = -bT(t)$.
  In this case the solution of the equation becomes then:

  \begin{align*}
    \frac{dT(t)}{dt} &= -bT(t) \\
    \frac{dT(t)}{dT(t)} &= -bdt \\
    \int \frac{q}{dT(t)}dt & \int -bdt \\
    \log(T(t)) &= -bt + C \\
    T(t) &= e^{-bt + C}\\
    T(t) &= C_2e^{-bt}
  \end{align*}

  So that the half-life then becomes:

  \begin{align*}
    T(t_1) &= \frac{1}{2}T(t_2)\\
    \cancel{C_2}e^{-bt_1} &= \frac{1}{2}\cancel{C_2}e^{-bt_2}\\
    2 &= \frac{e^{-bt_2}}{e^{-bt_1}}\\
    b(t_1-t_2) &= \log(2)\\
    t_1-t_2 &= \frac{\log(2)}{b}\\
    t_1-t_2 &= \frac{\log(2)}{1.9 \cdot 10^{-5}} = 3684.143\text{ years}\\
  \end{align*}



  \subsection{Calculate $\mathbf{T(t)}$, solution of the second differential equation}

  $$\frac{dT(t)}{dt} = aU_0e^{-at} - bT(t)$$

  Let:

  \begin{align*}
    w(t) &= \frac{T(t)}{e^{-bt}}\\
    w(t) &= T(t)e^{bt}
  \end{align*}

  Taking its derivative:

  \begin{align*}
    \frac{dw(t)}{dt} &= bT(t)e^{bt} + [aU_0e^{-at} - bT(t)]e^{bt}\\
    \frac{dw(t)}{dt} &= bT(t)e^{bt} + aU_0e^{-at + bt} - bT(t)e^{bt}\\
    \frac{dw(t)}{dt} &= \cancel{bT(t)e^{bt}} + aU_0e^{-at + bt} \cancel{- bT(t)e^{bt}}\\
    \frac{dw(t)}{dt} &= aU_0e^{-at + bt}
  \end{align*}

  Now taking the integral:

  \begin{align*}
    \int\frac{dw(t)}{dt}dt &= \int aU_0 e^{-at + bt}dt\\
    w(t) = \int aU_0\int e^{(b-a)t}dt\\
    w(t) = \frac{aU_0}{b-a}e^{b-at} + C_2
  \end{align*}

  Now, considering that:

  $$T(t) = w(t)e^{-bt}$$

  \begin{align*}
    T(t) & = \frac{aU_0}{b-a}e^{(b-a)t}e^{-bt} + C_2e^{-bt}\\
    T(t) & = \frac{aU_0}{b-a}\cancel{e^{bt}}e^{-at}\cancel{e^{-bt}} + C_2e^{-bt}\\
         & = \frac{aU_0}{b-a}e^{-at} + C_2e^{-bt}
  \end{align*}

  Considering $T(0) = 0$:

  \begin{align*}
    \frac{aU_0}{b-a}e^{-at} + C_2e^{-bt} & = 0\\
    \frac{aU_0}{b-a} + C_2 & = 0\\
    C_2 &= \frac{-aU_0}{b-a}\\
  \end{align*}

  So, in the end:

  $$T(t) = \frac{aU_0}{b-a}e^{-at} - \frac{aU_0}{b-a}e^{-bt}$$

  \subsection{Compute $\mathbf{\lim\limits_{t \to \infty} \frac{T(t)}{U(t)}}$}

  Computing the ratio:

  \begin{align*}
     \frac{T(t)}{U(t)} &= \frac{\frac{aU_0}{b-a}e^{-at} - \frac{aU_0}{b-a}e^{-bt}}{U_0e^{-at}}\\
                       &= \frac{aU_0}{b-a}e^{-at}\frac{1}{U_0e^{_at}} - \frac{aU_0}{b-a}e^{-bt}\frac{1}{U_0e^{_at}}\\
                       &= \frac{a\cancel{U_0}}{b-a}\cancel{e^{-at}}\frac{1}{\cancel{U_0}\cancel{e^{_at}}} - \frac{a\cancel{U_0}}{b-a}e^{-bt}\frac{1}{\cancel{U_0}e^{-at}}\\
                       &= \frac{a}{b-a} - \frac{a}{b-a}e^{(a-b)t}
   \end{align*}


  \begin{align*}
    \lim\limits_{t \to \infty} \frac{T(t)}{U(t)} = \lim\limits_{t\to\infty} \frac{a}{b-a} - \frac{a}{b-a}e^{(a-b)t} &= \frac{a}{b-a} - \frac{a}{b-a}e^{(a-b)\infty}\\
                                          &= \frac{a}{b-a} - \frac{a}{b-a}0\\
                                          &= \frac{a}{b-a}\\
  \end{align*}

  \subsection{Explain why it is possible to estimate the rock age from the knowledge of $\mathbf{\frac{T(t)}{U(t)}}$ at current time}
  \textit{Hint: write down the function $\frac{T(t)}{U(t)}$ and evaluate its behaviour.}
  \\
  \\
  Consider:

  $$\frac{T(t)}{U(t)} = \frac{a}{b-a} - \frac{a}{b-a}e^{(a-b)}$$

  Now, this function does not depend on the initial concentration of Uranium and can be inverted for $t$:

  \begin{align*}
    \frac{T(t)}{U(t)} &= \frac{a}{b-a} - \frac{a}{b-a}e^{(a-b)t}\\
    \frac{a}{b-a}e^{(a-b)t} &= \frac{a}{b-a} - \frac{T(t)}{U(t)}  \\
    e^{(a-b)t} &= \cancel{\frac{a}{b-a}}\cancel{\frac{b-a}{a}} - \frac{b-a}{a}\frac{T(t)}{U(t)}  \\
    e^{(a-b)t} &= 1 - \frac{b-a}{a}\frac{T(t)}{U(t)}\\
    (a-b)t &= \log\left(1 - \frac{b-a}{a}\frac{T(t)}{U(t)}\right)\\
    t\left(\frac{T(t)}{U(t)}\right) &= \frac{\log\left(1 - \frac{b-a}{a}\frac{T(t)}{U(t)}\right)}{a-b}
  \end{align*}

  So now from the ratio $\frac{T(t)}{U(t)}$ the rock age can be estimated.

\section{Problem 4}
It may be considered reasonable that, for a sexual species, births are proportional to the number of encounters, hence, disregarding the mortality process:

$$\frac{dN(t)}{dt} = \beta N^2(t)$$

  \subsection{This equation can be solved by the method of separation of variables. Show that the solutions of this equation with $\mathbf{N(0) > 0}$ tend to infinity in a finite time}
  Solving the equation through separation of variables:

  \begin{align*}
    \frac{dN(t)}{dt} &= \beta N^2(t)\\
    \frac{1}{N^2}dN(t) &= \beta dt\\
    \int\frac{1}{N^2}dN(t) &= \int\beta dt\\
    -\frac{1}{N(t)} &= \beta t + C\\
    N(t) = \frac{1}{C-\beta t}
  \end{align*}

  This equation has an accumulation point at $t = \frac{C}{\beta}$, so let's compute the limit of it approaching from $0$:

  \begin{align*}
    \lim\limits_{t\to\frac{C}{\beta}^-} &= \frac{1}{C-\cancel{\beta}\frac{C}{\cancel{\beta}}^-}\\
                                        &=\frac{1}{C-C^-}\\
                                        &=\frac{1}{0^+}\\
                                        &=+\infty
  \end{align*}


  \subsection{Let us correct the equation, introducing deaths: $\mathbf{\frac{dN(t)}{dt} = \beta N^2(t)-\mu N(t)}$. Discuss how the dynamics changes. In particular, does the equation still have the problem of solutions going to infinity in a finite time?}
  \textit{One may answer this question by computing the exact solutions of the equation. It is recommended instead to base the answer on a qualitative argument, comparing the two}\\
  Now the growth has a limiting factor represented by the death.
  The problem is that the number of death grows with an order of magnitude less with respect to the births, so the solution still goes to infinity in finite time.
  In particular the new equation is not monotonic anymore looking at the sign of the derivative:


  \begin{figure}[H]
    \centering
    \begin{tikzpicture}
      %interval
      \draw[-latex] (-2,0)--(5,0);
      %Vertical lines
      \draw (0,-2)--(0,.1);
      \node[above] at (0,0.1) () {\small 0};
      \draw (3,-2)--(3,.1);
      \node[above] at (3,0.1) () {\small $\frac{\mu}{\beta}$};
      \draw (-2,-1.5)--(5,-1.5);

      %First solution
      \node at (-3,-.5) {$N(t)\ge 0$};
      \node at (-1,-.5) () {-};
      \node at (1.5,-.5) () {+};
      \node at (4,-.5) () {+};

      %Second solution
      \node at (-3,-1) {$N(t) \ge \frac{\mu}{\beta}$};
      \node at (-1,-1) () {-};
      \node at (1.5,-1) () {-};
      \node at (4,-1) () {+};

      %Final solution
      \node at (-1,-2) {+};
      \node at (1.5,-2) () {-};
      \node at (4,-2) () {+};
    \end{tikzpicture}
  \end{figure}

  It will reach a minimum at $\frac{\mu}{\beta}$ and then grow again to infinity.
  This is the point where the deaths are unable to keep up with the births and the population begins to grow.


  \subsection{Let us introduce a limiting factor of logistic type:\\ $\mathbf{\frac{dN(t)}{dt} = \left(\beta N^2(t) - \mu N(t)\right)(1-\gamma N(t))}$ Find all non-negative equilibria and discuss their stability. From the direction field conclude that all solutions of are bounded}
  \textit{Hence, the theory of differential equations ensure that solutions are global, i.e. defined for all $t > 0$.}


  To find the equilibria:

  \begin{align*}
    \left(\beta N^2(t) - \mu N(t)\right)(1-\gamma N(t)) &= 0\\
    N(t)\left(\beta N(t) - \mu\right)(1-\gamma N(t)) &= 0\\
    N(t) = 0 \qquad \lor\qquad N(t) = \frac{\mu}{\beta}\qquad\lor\qquad N(t) = \frac{1}{\gamma}
  \end{align*}

  Now, computing their stability.
  First compute the derivative of the second-hand of the differential equation:

  \begin{align*}
    \frac{df(N(t), t)}{dN(t)} &= \frac{d}{dN(t)}\left(\beta N^2(t) - \mu N(t)\right)(1-\gamma N(t))\\
                              &=\frac{d}{dN(t)}\left(\beta N^2(t)-\mu N^2(t)-\beta\gamma N^3(t)+\mu\gamma N^2(t)\right)\\
                              &=-3\beta\gamma N^2(t) + (2\beta + 2\mu\gamma)N(t) - \mu
  \end{align*}

  Now let's consider $N(t) = 0$:

  \begin{align*}
    \frac{df(0, t)}{dN(t)} &= -3\beta\gamma 0 + (2\beta + 2\mu\gamma)0 - \mu\\
                           &= -\mu
  \end{align*}

  Since $\mu\ge 0$ $\frac{df(0, t)}{dN(t)} <0$, so the equilibrium is asymptotically stable.\\
  Now let's consider $N(t) = \frac{\mu}{\beta}$:

  \begin{align*}
    \frac{df(\frac{\mu}{\beta}, t)}{dN(t)} &= -3\beta\gamma\left(\frac{\mu}{\beta}\right)^2 + (2\beta + 2\mu\gamma)\frac{\mu}{\beta} - \mu\\
                                           &=-\frac{3\gamma\mu^2}{\beta} + 2\mu+2\gamma\frac{\mu^2}{\beta}-\mu\\
                                           &=-3\beta\gamma\left(\frac{\mu}{\beta}\right)^2+\beta\frac{\mu}{\beta}+2\gamma\beta\left(\frac{\mu}{\beta}^2\right)\\
                                           &=-\beta\gamma\left(\frac{\mu}{\beta}\right)^2 + \beta\frac{\mu}{\beta}\\
                                           &=\frac{\mu}{\beta}\left(-\beta\gamma\frac{\mu}{\beta} + \beta\right)
  \end{align*}

  Now let's discuss the sign:

  \begin{align*}
    \frac{\mu}{\beta}\left(-\cancel{\beta}\gamma\frac{\mu}{\beta} + \cancel{\beta}\frac{\mu}{\beta}\right) &\ge 0
    \frac{\mu}{\beta} \ge 0\qquad \frac{\mu}{\beta} \le \frac{1}{\gamma} \\
  \end{align*}

  Since $\mu\ge 0$ $\frac{df(0, t)}{dN(t)} <0$, so the equilibrium is asymptotically stable.\\

  Now, solving the inequality:

  \begin{figure}[H]
    \centering
    \begin{tikzpicture}
      %interval
      \draw[-latex] (-2,0)--(5,0);
      %Vertical lines
      \draw (0,-2)--(0,.1);
      \node[above] at (0,0.1) () {\small 0};
      \draw (3,-2)--(3,.1);
      \node[above] at (3,0.1) () {\small $\frac{1}{\gamma}$};
      \draw (-2,-1.5)--(5,-1.5);

      %First solution
      \node at (-3,-.5) {$\frac{\mu}{\beta}\ge 0$};
      \node at (-1,-.5) () {-};
      \node at (1.5,-.5) () {+};
      \node at (4,-.5) () {+};

      %Second solution
      \node at (-3,-1) {$\frac{\mu}{\beta} \le \frac{1}{\gamma}$};
      \node at (-1,-1) () {+};
      \node at (1.5,-1) () {+};
      \node at (4,-1) () {-};

      %Final solution
      \node at (-1,-2) {-};
      \node at (1.5,-2) () {+};
      \node at (4,-2) () {-};
    \end{tikzpicture}
  \end{figure}

  So the equilibrium $\frac{\mu}{\beta}$ is asymptotically stable if it is $\le 0$ (always true) or if it is $\ge\frac{1}{\gamma}$.
  Otherwise it is unstable.\\
  Now let's consider $N(t) = \frac{1}{\gamma}$.

  \begin{align*}
    \frac{df(\frac{\mu}{\beta}, t)}{dN(t)} &= -3\beta\gamma\left(\frac{1}{\gamma}\right)^2 + (2\beta + 2\mu\gamma)\frac{1}{\gamma} - \mu\\
                                           &=-3\beta\frac{1}{\gamma} + 2\beta\frac{1}{\gamma} + 2\mu -\mu\\
                                           &=-\beta\frac{1}{\gamma} +\mu\\
  \end{align*}

  Now solving the inequality:

  \begin{align*}
    -\beta\frac{1}{\gamma} +\mu \ge 0\\
    \frac{1}{\gamma}\le \frac{\mu}{\beta}
  \end{align*}

  So the equilibrium is asymptotically stable if it $\ge\frac{\mu}{\beta}$, otherwise it is unstable.\\
  This means that the biggest equilibrium is always asympotically stable, effectively posing a bound on the solution of the equation.\\
  Drawing the direction field using Matlab:

  \begin{lstlisting}[language=Matlab]
    [T, N] = meshgrid(-2:0.2:4, -2:0.2:4);
    b = 1; m = 2; g = 1;
    S = (b.*N.^2-m.*N).*(1-g.*N);
    L = sqrt(1 + S.^2);
    quiver(T,N, 1./L,S./L, 0.5), axis tight
  \end{lstlisting}

  \begin{figure}[H]
    \centering
    \includegraphics[width=\textwidth/2]{04_direction_field}
  \end{figure}
