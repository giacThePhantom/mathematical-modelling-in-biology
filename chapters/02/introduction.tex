\chapter{Biological models based on differential equations}

\section{The bathtub}
Let $Q(t)$ the quantity of a substance in the bathtub, then:

$$Q'(t) = I(t) - O(t)$$

Where:

\begin{multicols}{2}
	\begin{itemize}
		\item $I(t)$ is the input rate: the quantity that enters in the time unit.
		\item $O(t)$ is the output rate, the quantity that exits in the time unit.
	\end{itemize}
\end{multicols}

If $I_{t, t+\delta t}$ is the quantity that enters in the interval, then $I_{t, t+\delta t} = I(t)\Delta t + o(\Delta t)$, where $o(\Delta t)$ is a higher order infinitesimal than $\Delta t$.
This means that when $\Delta t$ goes to $i$, not only $\lim\limits_{\Delta t \rightarrow 0}o(\Delta t) = 0$ and that $\lim\limits_{\Delta t\rightarrow 0}\frac{o(\Delta t)}{\Delta t} = 0$.
Hence:

$$I(t) = \lim\limits_{\Delta t\rightarrow 0}\frac{I_{t, t+\Delta t}}{\Delta t}$$

So the input rate behaves like an instantaneous velocity and is measured in $[\frac{C}{t}]$ Where $[C]$ represents the concentration.
The same applies for the exit rate $O(t)$.

	\subsection{An example}
	Let's assume that $I(t) = \Lambda$, a constant input flux and $O(t) = \gamma Q(t)$, the exit flux is proportional to the quantity present at the moment.
	The proportionality constant $\gamma$ is often called the exit rate and has dimension $[t^{-1}]$.
	From these assumptions:

	$$Q'(t) = \Lambda - \gamma Q(t)$$

	To which the initial condition $Q(0) = Q_0$ is added.
	The solution to the equation is:

	$$Q(t) = e^{-\gamma t}Q_0+\frac{\Lambda}{\gamma}(1-e^{-\gamma t})$$

	If $\Lambda = 0$ the solution is then $Q(t) = Q_0e^{-\gamma t}$, meaning that the survival time of a molecule initially present follows the exponential distribution:

	$$\mathbb{P}(\text{a molecule present at time }0\text{ is present at time }t>0) = e^{-\gamma t}$$

	So the mean survival time $\mathbb{E}(T) = \frac{1}{\gamma}$ and the exit rate $\gamma$ can be interpreted as the inverse of the mean survival time.

\section{Malthus equation}
The bathtub can be used to model the dynamics of a population.
Neglecting all differences among individuals a population can be represented through its size $N(t)$.
This will increase through input due to births and output due to death, hence:

$$N'(t) = B(t)-D(t)$$

Where:

\begin{multicols}{2}
	\begin{itemize}
		\item $B(t)$ are the births.
		\item $D(t)$ are the deaths.
	\end{itemize}
\end{multicols}
