\input{prefix}

\title{\Huge\textbf{{Mathematical modelling in biology}}\\\LARGE Some models}

\author{
  Giacomo Fantoni \\
  \small telegram: \href{https://t.me/GiacomoFantoni}{@GiacomoFantoni} \\[3pt]
  \small Github: \href{https://github.com/giacThePhantom/mathematical-modelling-in-biology}{https://github.com/riacchiappando/mathematical-modelling-in-biology}\\
}


\begin{document}

  \maketitle
  \tableofcontents

\section{The bathtub}
Imagine a container in which there exists an input of water $I(t)$ and an output $O(t)$.
Let $V(t)$ be the volume of water in the bathtub.
The varation in time of the volume of water is:

$$\frac{dV(t)}{dt} = I(t)-O(t)$$

  \subsection{Constant input}
  Assume that the input is constant $I(t) = \Lambda$ and that the output depends on $V(t)$ through a constant $\gamma$.
  The problem then becomes:

  $$\frac{dV(t)}{dt} = \Lambda - \gamma V(t)$$

  This can be solved thorugh the variation of constants method.
  First solve the associated homogeneous equation:

  \begin{align*}
    \frac{du(t)}{dt} = -\gamma u(t)\\
    \frac{du(t)}{u(t)} = -\gamma dt\\
    \int\frac{du(t)}{u(t)} = -\gamma\int dt\\
    \ln(u(t)) = -\gamma t + c\\
    u(t) = e^{-\gamma t + c}\\
    u(t) = Ce^{-\gamma t}\\
  \end{align*}

  Then we can write $V(t) = C(t)e^{-\gamma t}$.
  Computing its derivative:

  $$\frac{dV(t)}{dt} = \frac{dC(t)}{dt}e^{-\gamma t} - \gamma C(t)e^{-\gamma t}$$

  Posing it equal to the starting one:

  \begin{align*}
    \Lambda - \gamma V(t) &= \frac{dC(t)}{dt}e^{-\gamma t} -\gamma C(t)e^{-\gamma t}\\
    \Lambda - \cancel{\gamma V(t)} &= \frac{dC(t)}{dt}e^{-\gamma t} -\cancel{\gamma V(t)}\\
    \frac{dC(t)}{dt} &= \Lambda e^{\gamma t}\\
    C(t) &= \int\Lambda e^{\gamma t}dt\\
    C(t) &= \frac{\Lambda}{\gamma}e^{\gamma t} + c\\
  \end{align*}

  So that the solution then becomes:

  \begin{align*}
    V(t) &= \left[\Lambda\frac{e^{\gamma t}{\gamma} + c\right]e^{-\gamma t}}\\
         &= \frac{\Lambda}{\gamma} + ce^{-\gamma t}\\
  \end{align*}

  The same thing can be done using an integrating factor:

  \begin{align*}
    \frac{dV(t)}{dt} &= \Lambda - \gamma V(t)\\
    \frac{dV(t)}{dt} + \gamma V(t) &= \Lambda\\
    e^{\int \gamma dt}\frac{dV(t)}{dt} + \gamma e^{\int\gamma dt}V(t) &= \Lambda e^{\int\gamma dt}\\
    e^{\gamma t}\frac{dV(t)}{dt} + \gamma e^{\gamma t}V(t) &= \Lambda e^{\gamma t}\\
    \frac{d}{dt}\left[e^{\gamma t}V(t)\right] &= \Lambda e^{\gamma t}\\
    e^{\gamma t}V(t) &= \int\Lambda e^{\gamma t}dt\\
    e^{\gamma t}V(t) &= \frac{\Lambda}{\gamma}e^{\gamma t} + c\\
    V(t) &= \frac{\Lambda}{\gamma} + ce^{-\gamma t}\\
  \end{align*}


  \subsection{Constant input and no output}
  In the case in which there is no output the problem then becomes:

  $$\frac{dV(t)}{dt} = \Lambda$$

  This is easy to solve:

  $$V(t) = \Lambda t + c$$

  \subsection{Varying input}
  Let now the input be a function of time $I(t) = \Lambda(t)$.
  The equation becomes:

  $$\frac{dV(t)}{dt} = \Lambda(t)$$

  The solutions of this is found by integrating both sides:

  $$V(t) + \int_0^t\Lambda(u)du + c$$

  \subsection{Output flux but no input}
  Let now the input be a function of time $O(t) = \gamma V(t)$.
  The variation in time becomes:

  $$\frac{dV(t)}{dt} = -\gamma V(t)$$

  This does not explicitly depends on $t$: it is autonomous.
  This can be solved thourgh the separation of variable methods:

  \begin{align*}
    \frac{dV(t)}{dt} &= -\gamma V(t)\\
    \frac{dV(t)}{V(t)} &= -\gamma dt\\
    \int\frac{dV(t)}{V(t)} &= -\gamma\int dt\\
    \ln(V(t)) &= -\gamma t + c\\
    V(t) &= e^{-\gamma t + c}\\
    V(t) &= e^{-\gamma t}e^c\\
    V(t) &= kre^{-\gamma t}
  \end{align*}

  $k$ in this context is the volume in the bathtub at time $0$, wich could be computed if the volme at time $0$ was given:

  $$\begin{cases}
    \frac{dV(t)}{dt} = -\gamma V(t)\\
    V(0) = V_0
  \end{cases}$$

  So that $V(0)$ can be computed:

  \begin{align*}
    V(0) &= kre^{-\gamma 0}\\
    V(0) &= k
  \end{align*}

\section{Malthus equation}
The Malthus equation is a model for the growth of a population.
It neglets difference amond individuals and migrations.
It represents a population through its size that will increase thru reproduction and decrease through death:

$$\frac{dN(t)}{dt} = B(t) - D(t)$$

The number of births can deaths are linked to the current population.
A death rate $\mu$ can be introduced ($\frac{1}{\mu}$ is the average lifespan) and a birth rate $\beta$ (the average number of newborn generated during a lifespan).
Both are non-negative constants:

\begin{align*}
  \frac{dN(t)}{dt} = \beta N(t) - \mu N(t)\\
  \frac{dN(t)}{dt} = (\beta - \mu)N(t)\\
  \frac{dN(t)}{dt} = rN(t)
\end{align*}

Where $r = \beta-\mu$ is the instanteous growth rate or Malthus parameter or biological potential of the population.
The equation can be solved through the separation of variables method:

\begin{align*}
  \frac{dN(t)}{dt} = rN(t)\\
  \frac{dN(t)}{N(t)} = rdt\\
  \int\frac{1}{N(t)}dN(t) = \int rdt\\
  \ln(N(t)) = rt + c\\
  N(t) = e^{rt + c}\\
  N(t) = ke^{rt}\\
  N(t) = ke^{(\beta-\mu)t}\\
\end{align*}

Where $k$ is the population size at time $0$.
If $r<0$ the population will go extint, while if $r>0$ it will grow exponentially.
If $r=0$ the population is constant.
The basic reproduction number $R_0=\frac{\beta}{\mu}$ can be considered.
$r<0$ is equivalent to $R_0<1$, while $R_0>1$ is equivalent to $r>0$.
Now, the equilibria and stability can be computed:

\begin{align*}
  rN(t) &= 0\\
  N(t) &= 0
\end{align*}

This is the only equilibrium point.

\section{The logistic equation}
The logistic equation introduces into the Malthus' one a term that limits the growth of the population.
The simplest way to do so is to modify the rates, supposing that fertility decreses and mortailtiy inreases inearly with $N(t)$:

\begin{align*}
  \beta(N(t)) = \beta_0 - \tilde{\beta}N(t)\\
  \mu(N(t)) = \mu_0 + \tilde{\mu}N(t)
\end{align*}

Where $\beta_0, \tilde{\beta}, \mu_0, \tilde{\mu}$ are positive constants.
Now the equation becomes:

\begin{align*}
  \frac{dN(t)}{dt} &= \beta(N(t))N(t) - \mu(N(t))N(t)\\
                   &= (\beta_0 - \tilde{\beta}N(t))N(t) - (\mu_0 + \tilde{\mu}N(t))N(t)\\
                   &= \beta_0N(t) - \tilde{\beta}N^2(t) - \mu_0N(t) - \tilde{\mu}N^2(t)\\
                   &= N(t)\left[\beta_0 - \tilde{\beta}N(t) - \mu_0 - \tilde{\mu}N(t)\right]\\
                   &= N(t)\left[(\beta_0 - \mu_0) - (\tilde{\beta} - \tilde{\mu})N(t)\right]\\
                   &= N(t)(\beta_0-\mu_0)\left[1-\frac{N(t)(\tilde{\beta}-\tilde{\mu})}{\beta_0-\mu_0}\right]\\
\end{align*}

Now $(\beta_0-\mu_0) = r$, the Malthus parameter, and $\frac{(\beta_0-\mu_0)}{(\tilde{\beta}-\tilde{\mu})} = K$, the carrying capacity:


$$\frac{dN(t)}{dt} = rN(t)\left[1-\frac{N(t)}{K}\right]$$

If both $\tilde{\beta}$ and $\tilde{\mu}$ are $0$ the equation goes back to the Malthus one.
For very large $N(t)$ and $\tilde{\beta}>0$, the birth rate could go negative.
This does not cause mathamtical problems and the conditions are which it happens do not occur, so it is neglected.
Generally it is assumed that $r>0$, since the population is growing over time.
So if $r>0$, and considering all the other assumptions $K>0$.

  \subsection{Solution}
  To solve the equation we solve for its reciprocal:

  $$u(t) = \frac{1}{N(t)}$$

  This implies that:

  \begin{align*}
    \frac{du(t)}{dt} &= -\frac{1}{N^2(t)}\frac{dN(t)}{dt}\\
                     &= \frac{-rN(t)\left[1-\frac{N(t)}{K}\right]}{N^2(t)}\\
                     &= \frac{-r\left[1-\frac{N(t)}{K}\right]}{N(t)}\\
                     &= -r\left[\frac{1}{N(t)}-\frac{1}{K}\right]\\
                     &= -ru(t) + \frac{r}{K}\\
  \end{align*}

  This is a linear non-homoeneous differential euqation that can be solved with the variation of constants method.
  Solving first the associated homogeneous problem thorugh sepration of variables:

  \begin{align*}
    \frac{du(t)}{dt} &= -ru(t)\\
    \frac{du(t)}{u(t)} &= -rdt\\
    \int\frac{1}{u(t)}du(t) &= \int-rdt\\
    \ln(u(t)) &= -rt + c\\
    u(t) &= e^{-rt+c}\\
    u(t) &= Ce^{-rt}\\
  \end{align*}

  Let $C$ be a function of $t$:

  $$u(t) = C(t)e^{-rt}$$

  And derive it:

  $$\frac{du(t)}{dt} = C'(t)e^{-rt} - rC(t)e^{-rt}$$

  Considering that:

  $$\frac{du(t)}{dt} = -ru(t)+\frac{r}{K}$$

  So that:

  \begin{align*}
    -ru(t) + \frac{r}{K} &= \frac{dC(t)}{dt}e^{-rt} - rC(t)e^{-rt}\\
    -\cancel{rC(t)e^{-rt}}+ \frac{r}{L} &= \frac{dC(t)}{dt}e^{-rt} - \cancel{rC(t)e^{-rt}}\\
    \frac{dC(t)}{dt}e^{-rt} &= \frac{r}{K}\\
    \frac{dC(t)}{dt} &= \frac{r}{K}e^{rt}\\
    C(t) &= \frac{r}{K}\int e^{rt}dt\\
    C(t) &= \frac{\cancel{r}}{K\cancel{r}}e^{rt} + c\\
    C(t) = \frac{1}{K}e^{rt} + c\\
  \end{align*}

  Subsitituting back into $u(t)$:

  \begin{align*}
    u(t) &= \lefft[\frac{e^{rt}}{K} + c\right]e^{-rt}\\
    u(t) = \frac{1}{K} + ce^{-rt}\\
  \end{align*}

  So that:

  $$N(t) = \frac{1}{\frac{1}{K} + ce^{-rt}}$$

  Now, solving the Cauchy problem where $N(0) = N_0$:

  \begin{align*}
    \frac{1}{\frac{1}{K} + ce^{-r0}} &= N_0\\
    \frac{1}{\frac{N_0}{K} + cN_0} &= 0\\
    \frac{N_0}{K} + cN_0 &= 1\\
    cN_0 &= 1 - \frac{N_0}{K}\\
    c &= \frac{1}{N_0} - \frac{1}{K}\\
  \end{align*}

  So that, substituting:

  \begin{align*}
    N(t) &= \frac{1}{\frac{1}{K} + \left[\frac{1}{N_0} -\frac{1}{K}\right]e^{-rt}}\\
         &= \frac{K}{1 + \left[\frac{K-N_0}{N_0}\right]e^{-rt}}\\
  \end{align*}

  \subsection{Equilibria and stability}
  Assume $r>0$ and $K>0$:

  \begin{align*}
    rN(t)\left[1-\frac{N(t)}{K}\right] &= 0\\
    N(t) = 0\qquad\land\qquad N(t) &= K\\
  \end{align*}

  The first equilibrium is unstable, since the population is growing, while the second is stable.


\section{Logistic equation with periodic carrying capacity}
The logistic model can be modified by assomng that the carrying capacity $K$ is a periodic, always positive function of $t$.
For example:

$$k(t) = K_0(1+\epsilon\cos(\omega t))\qquad\qquad 0<\epsilon< 1$$

The model then becomes:

\begin{align*}
  \frac{dN(t)}{dt} &= rN(t)\left[1-\frac{N(t)}{K(t)}\right]\\
                   &= rN(t)\left[1-\frac{N(t)}{K_0(1+\epsilon\cos(\omega t))}\right]\\
\end{align*}

Which makes the equation not autonomous anymore.

  \subsection{Solution}
  The equation can be solved with the same reciprocal trick:

  $$u(t) = \frac{1}{N(t)}$$

  \begin{align*}
    \frac{du(t)}{dt} &= -\frac{1}{N^2(t)}\frac{dN(t)}{dt}\\
                     &= \frac{-r\cancel{N(t)}\left[1-\frac{N(t)}{K(t)}\right]}{N^{\cancel{2}(t)}}\\
                     &= \frac{-r\left[1-\frac{N(t)}{K(t)}\right]}{N(t)}\\
                     &= -r\left[\frac{1}{N(t)}-\frac{1}{K(t)}\right]\\
                     &= -ru(t) + \frac{r}{K(t)}\\
  \end{align*}

  Now this can be solved with the Variation of constants.
  Solving the associated homogeneous problem:

  $$u(t) = Ce^{-rt}$$

  Now let $C$ a function of $t$ and compute the derivative:

  $$\frac{du(t)}{dt} = \frac{dC(t)}{dt}e^{-rt} - rC(t)e^{-rt}$$

  So now, substituting what we now:

  \begin{align*}
    \frac{dC(t)}{dt}e^{-rt} - rC(t)e^{-rt} = -ru(t) + \frac{r}{K(t)}\\
    \frac{dC(t)}{dt}e^{-rt} - rC(t)e^{-rt} = -ru(t) + \frac{r}{K(t)}\\
    \frac{dC(t)}{dt}e^{-rt} - \cancel{rC(t)e^{-rt}} = -\cancel{rC(t)e^{-rt}} + \frac{r}{K(t)}\\
    \frac{dC(t)}{dt}e^{-rt} = \frac{r}{K(t)}\\
    \frac{dC(t)}{dt} = \frac{r}{K(t)}e^{rt}\\
    C(t) = r\int \frac{e^{rt}}{K(t)}dt\\
    C(t) = r\int \frac{e^{rt}}{K_0(1+\epsilon\cos(\omega t))}dt\\
    C(t) = \frac{r}{K_0}\int \frac{e^{rt}}{1+\epsilon\cos(\omega t)}dt\\
  \end{align*}

  Which has no analyitical solution.
  So now:

  $$u(t) = e^{-rt}\frac{r}{K_0}\int \frac{e^{rt}}{1+\epsilon\cos(\omega t)}dt$$

  And:

  $$N(t) = \frac{e^{rt}K_0}{r\int \frac{e^{rt}}{1+\epsilon\cos(\omega t)}dt}$$

  \subsection{Equilibria}
  Assuming still $r>0$ and $K_0>0$, the equilibria can be computed:

  \begin{align*}
    rN(t)\left[1-\frac{N(t)}{K(t)}\right] &= 0\\
    N(t) = 0\qquad\land\qquad N(t) &= K(t)\\
  \end{align*}

  The first equlibria is unstable.
  The second is an asymptotically stable periodic equilibria with period $T = \frac{2\pi}{\omega}$.

\end{document}
